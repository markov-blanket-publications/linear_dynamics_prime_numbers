\documentclass{article}

% ready for submission
\usepackage{arxiv}

\usepackage[utf8]{inputenc} % allow utf-8 input
\usepackage[T1]{fontenc}    % use 8-bit T1 fonts
\usepackage{hyperref}       % hyperlinks
\usepackage{url}            % simple URL typesetting
\usepackage{booktabs}       % professional-quality tables
\usepackage{amsfonts}       % blackboard math symbols
\usepackage{nicefrac}       % compact symbols for 1/2, etc.
\usepackage{microtype}      % microtypography
\usepackage{amsmath}}

\title{A linear dynamical system identified with optimal prime sieving}

\date{February 25, 2021}

% The \author macro works with any number of authors. There are two commands
% used to separate the names and addresses of multiple authors: \And and \AND.
%
% Using \And between authors leaves it to LaTeX to determine where to break the
% lines. Using \AND forces a line break at that point. So, if LaTeX puts 3 of 4
% authors names on the first line, and the last on the second line, try using
% \AND instead of \And before the third author name.

\author{%
  Aidan Rocke\\
  \texttt{aidanrocke@gmail.com} \\
  % examples of more authors
  % \And
  % Coauthor \\
  % Affiliation \\
  % Address \\
  % \texttt{email} \\
  % \AND
  % Coauthor \\
  % Affiliation \\
  % Address \\
  % \texttt{email} \\
  % \And
  % Coauthor \\
  % Affiliation \\
  % Address \\
  % \texttt{email} \\
  % \And
  % Coauthor \\
  % Affiliation \\
  % Address \\
  % \texttt{email} \\
}

\begin{document}

\maketitle

\begin{abstract}
There exists a linear dynamical system that may be identified with an optimal prime-sieve. This
dynamical system is interesting because its phase-space dimension is unbounded. From this we may deduce that the
Kolmogorov Complexity of the Koopman operator is unbounded. This means that the shortest program in any language
that could give the nth prime does not have finite length. More importantly, given that verifying the correctness of such a program would scale with program length, even if such a program existed we would not be able to verify its correctness in a finite number of steps.
	
\end{abstract}

\section{An optimal prime-sieve}

The goal of an optimal prime sieve is to find all prime numbers less than or equal to $N \in \mathbb{N}$.
A relatively simple solution for finding the first $n$ prime numbers $\mathbb{P}^n = \{p_i\}_{i=1}^n$
is given by the following program.

First, define the function $f$ that takes as inputs $N \in \mathbb{N}$ and $\mathbb{P}^n = \{p_i\}_{i=1}^n$
and computes:

\begin{equation}
f(N, \mathbb{P}^n) = \prod_{i=1}^n (N \mod p_i)
\end{equation}

If $f(N, \mathbb{P}^n) = 0$,

\begin{equation}
N:=N+1
\end{equation}

Otherwise, if $f(N, \mathbb{P}^n) = 1$, we execute the updates:

\begin{equation}
p_{n+1} = N
\end{equation}

\begin{equation}
\mathbb{P}^{n+1} := p_{n+1} \cup \mathbb{P}^n
\end{equation}

\begin{equation}
N := N+1
\end{equation}

and we halt this process when the cardinality of $\mathbb{P}^n$ is as desired.

\newpage

\section{The Buckingham-Pi theorem}

The Buckingham-Pi theorem states that if a dynamical system involves $n$ physical variables and $n$ also happens
to be the number of physical dimensions involved then you need a system of scientific units whose cardinality is
greater than or equal to $n$.

If you have a system of fundamental units $U = \{u_i\}_{i=1}^n$ and all the other physical units $C$ are derived
from $U$:

\begin{equation}
\forall c \in C, \exists \alpha_i \in \mathbb{Z}, c = \prod_{i=1}^n = u_i^{\alpha_i}
\end{equation}

Furthermore, if we define the n-dimensional vector space:

\begin{equation}
\text{span}(\log U) = \big\{ \sum_{i=1}^n \alpha_i \log u_i \lvert u_i \in U, \alpha_i \in \mathbb{Z} \big\}
\end{equation}

the $u_i$ are fundamental if they are dimensionally independent in the sense that $\forall u_i, u_{j \neq i} \in U$,
$\log u_i \perp \log u_{j \neq i}$:

\begin{equation}
\exists \alpha_i \in \mathbb{Z}, \sum_{i=1}^n \alpha_i \log u_i \iff \alpha_i = 1 \land \alpha_{j \neq i} = 0
\end{equation}

Moreover, it can be shown that:

\begin{equation}
\log C = \text{span}(\log U)
\end{equation}

as every element in $C$ has a unique factorisation in terms of $U$.

\section{The Koopman operator}

In principle, any physical system may be modelled as a discrete dynamical system:

\begin{equation}
\forall k \in \mathbb{Z}, x_{k+1} = \Psi \circ x_k
\end{equation}

where, without loss of generality, $x_k \in S \subset \mathbb{R}^n$, $k$ is a discrete time index and $T:S \rightarrow S$ is a dynamic map. This representation is epistemologically sound as data collected from dynamical systems always comes in discrete-time samples.

Within this context, we may represent data as evaluations of functions of the state $x_k$, known as observables. In fact, if $g: S \rightarrow \mathbb{R}$ is an observable associated with the system then the collection of all such observables forms a vector space due to the Buckingham-Pi theorem.

Now, the Koopman operator $\Psi$ is a linear transform on this vector space given by:

\begin{equation}
\Psi g(x) = g \circ \Psi(x)
\end{equation}

which in a discrete setting implies:

\begin{equation}
\Psi g(x_k) = g \circ \Psi(x_k) = g(x_{k+1})
\end{equation}

where the linearity of the Koopman operator follows from the linearity of the composition operator:

\begin{equation}
\Psi \circ (g_1 + g_2)(x) = g_1 \circ \Psi(x) + g_2 \circ \Psi(x)
\end{equation}

So we may think of the Koopman operator as lifting dynamics of the state space to the space of observables.

Furthermore, we may make the key observation that if $\Psi$ is of rank $n$ then $\Psi$ describes the
evolution of a dynamical system with an n-dimensional phase-space.

\newpage 

\section{The prime numbers satisfy the criteria of Buckingham-Pi}

It may be shown that $\mathbb{P}$ forms a fundamental system of units in the sense of Buckingham-Pi since the
elements of $\log \mathbb{P}$ are dimensionally independent $\forall p_j, p_{i \neq j} \in \mathbb{P}, \log p_{i \neq j} \perp \log p_j$ in the sense that:

\begin{equation}
\exists \alpha_i \in \mathbb{Z}, \sum_{i=1}^\infty \alpha_i \log p_i = \log p_j \iff \alpha_j = 1 \land \alpha_{i \neq j} = 0
\end{equation}

Furthermore, if we define the infinite-dimensional vector space:

\begin{equation}
\text{span}(\log \mathbb{P}) = \Big\{\sum_{i=1}^\infty \alpha_i \log p_i < \infty \lvert p_i \in \mathbb{P}, \alpha_i \in \mathbb{Z}\Big\}
\end{equation}

and if we define the first $n$ primes, $\mathbb{P}^n = \{p_i\}_{i=1}^n$ then by definition:

\begin{equation}
\text{span}(\log \mathbb{P}^n) \subset \text{span}(\log \mathbb{P})
\end{equation}

Moreover, due to the unique factorisation of the integers, it may be shown that:

\begin{equation}
\text{span}(\log \mathbb{Q}_+) = \text{span}(\log \mathbb{P})
\end{equation}

\section{A linear dynamical system associated with optimal prime sieving}

Given the optimal prime-sieve that was introduced earlier, we may identify the first $n$ prime numbers $\mathbb{P}^n$
with the n-dimensional vector space:

\begin{equation}
\text{span}(\log \mathbb{P}^n) = \Big\{\sum_{i=1}^n \alpha_i \log p_i \cdot \vec{e_i}  \lvert p_i \in \mathbb{P}, \alpha_i \in \mathbb{N}\Big\}
\end{equation}

where $\vec{e_i}$ is the usual n-dimensional unit vector with components $\delta_{ij}$.

Now, given that the dynamics on the integers(our states) may be described by the linear model:

\begin{equation}
\forall n \in \mathbb{N}, f \circ n = n+1
\end{equation}

we may construct a bijective mapping $G$ which lifts dynamics from the state-space $\mathbb{N}$ to the space of observables $\text{span}(\log \mathbb{P}^n)$ such that:

\begin{equation}
\forall n \in \mathbb{N} \exists! X_n \in \text{span}(\log \mathbb{P}^n), X_n = G \circ n
\end{equation}

where $X = \{X_i\}_{i=1}^n, X_i = \alpha_i \cdot \log p_i$ assuming that:

\begin{equation}
\exists \alpha_i \in \mathbb{N}, \log n = \sum_{i=1}^n \alpha_i \cdot \log p_i
\end{equation}

Now, let's suppose that there exists a linear map $\Psi \in \mathbb{R}^{n \times n}$ such that:

\begin{equation}
X_{n+1} = \Psi \circ X_n
\end{equation}

Given the infinitude of primes, the phase-space associated with the evolution of $X_n$ is unbounded and by the Buckingham-Pi
theorem the Kolmogorov Complexity $K(\cdot)$ of $\Psi$ at the nth iteration will scale with:

\begin{equation}
K(\Psi) \sim \ln(\pi(n)) \sim \ln \big(\frac{n}{\ln n}\big)
\end{equation}

due to the Prime Number Theorem.

\newpage

\section{Analysis}

From this analysis we may deduce that the shortest program in any programming language that could give the nth prime for all $n \in \mathbb{N}$ would have unbounded length. Furthermore, given that verifying the correctness of this program would scale with program length, $ \sim \ln \big(\frac{n}{\ln n}\big)$, we may deduce that even if such a program existed we would not be able to verify its correctness in a finite number of steps. 

As an immediate consequence, while a single counter-example may be used to prove that the Riemann Hypothesis is false, we can’t prove that the Riemann Hypothesis is true.

\section*{References}

\small

[1] Bernard Koopman. Hamiltonian systems and Transformations in Hilbert Space.

[2] Steven L. Brunton. Notes on Koopman operator theory. 2019.

[3] Peter D. Grünwald. The Minimum Description Length Principle
. MIT Press. 2007.

[4] M. Li and P. Vitányi. An Introduction to Kolmogorov Complexity and Its Applications. Graduate Texts in Computer Science. Springer. 1997.

[5] Fine & Rosenberger. Number Theory: An Introduction Via the Distribution of Primes. 2007.
[6] Don Zagier. Newman’s short proof of the Prime Number Theorem. The American Mathematical Monthly, Vol. 104, No. 8 (Oct., 1997), pp. 705-708

\end{document}

\end{document}